\documentclass[11pt]{article}

\usepackage{fullpage}
\usepackage{graphicx}

\begin{document}

\title{Our Extension...}
\author{Istvan Darok, James Shipley, Prashant Gurung}

\maketitle

\section {Assembler Structure and Implementation}
Our implementation of the assembler follows the strategy of peforming two passes
over the source code. \\~\\
We created a Pair structure to store labels with their corresponding memory
addresses, and a Symbol\_table structure which holds an array of these Pairs.
The first pass reads the input file and populates this symbol table structure 
with all of the labels in the source code. \\~\\
Afterwards, the second pass is performed,
where each instruction from the array is assembled and saved to an array of 
unsigned integers. Once every instruction has been assembled, this array is
now written to a binary output file. \\~\\
Similar to our emulator implementation, we have a struct 
Assembler, which holds variables such as the populated symbol table and the
input file (which is stored as an array of strings). By using an assembler struct,
we can construct a set of modular functions that are called one after another in
main and so our main function is very short and outlines the structure of our
assembler very clearly.

\section{Extension Overview}
For the extension, we decided to implement an interactive mandelbrot and
julia set viewer using the SDL library. Originally, we planned on using ncurses
for our extension, however we quickly realised that the library would not be able
to facilitate the high level of graphics required by our project.
After some research, we agreed to use SDL instead as it provides the same 
functionality as ncurses with regards to user interaction (keypress and mouse)
but has a better graphics library to suit our needs. \\~\\
By using keys on the keyboard, the user can:
\begin{itemize}
    \item Move up, down, left or right.
    \item Zoom in and out.
    \item Increase / decrease detail  (with a cost to performance).
\end{itemize}

\section{Extension Design}
The first struct, Complex, simply represents a complex value as it contains two 
long doubles each representing the real and imaginary components of a complex 
number.
\\~\\
We then have a Mandelbrot struct, which contains:
\begin{itemize}
    \item Complex center and long double range together represent the area of the
    mandelbrot set that is being focused on/ displayed.
    \item Each complex structure in the 2D array "area" represents the complex
    value of each pixel on the screen.
    \item int maxIter represents the maximum number of iterations for which the 
    function will be applied to a complex number.
    \item Itermap, the 2D array of integers, represents the number of iterations
    required for the complex number at that point to diverge. If it does not diverge,
    it will be set to maxIter.
\end{itemize}
The Screen structure allows us to refer to what is being displayed, and
any inputs from the user. The struct is populated with pointers to the data 
types from the SDL library. SDL\_Window *window and SDL\_Renderer *renderer
are used to output our program. SDL\_Event *event is used to poll any key inputs
the user has made.
\\~\\
The main program follows a simple algorithm:
\begin{itemize}
    \item Allocate memory for two mandelbrot structs and the screen struct.
    \item Initialise the elements in the 2D array of mandelbrot and julia set to its
    corresponding complex number.
    \item For each element in the 2D array of complex numbers, compute up the 'maxIter' 
    iteration, checking each time if the complex number diverges. If at any point it
    does, change its corresponding iterMap value to the current iteration.
    \item Display the current state of the mandelbrot and julia set.  
    \item The program then enters a while loop, which will not break until the user has
    input the quit command 'q'. In this loop, the user can enter inputs, which
    will be polled to navigate around the mandelbrot set, zoom in and out, and
    change the max number of iterations to be performed.
\end{itemize}
Initially, we had a naive implementation to compute the mandelbrot set.
Translating the mandelbrot set only requires to for either a column or the row 
of the array to be recomputed. However, zooming in and out requires the entire 
mandelbrot set to be recomputed.
So when lots of zoom in and out were performed, the program ran slowly.
After researching ways to better compute the mandelbrot set, we came across the
Mariani-Silver algorithm.
\\~\\
This algorithm relies on the fact that if a border of a region has the same 
colour, then every pixel in the region also has the same colour. Thus, when
calculating the mandelbrot and julia set, we only have to look at the permimeter
of a region. If the pixels along the perimeter do not have the same colour,
then we can subdivide the region, and recursively compute on the divided 
regions. This possibly prevents computation on a whole region (apart from
the perimeter).
\\~\\
The main challenge was implementing multi-threading for our project. Initially,
we hoped to be able to divide the mandelbrot and julia set into several regions,
with a thread working on each region. We however quickly realised that it would
be very difficult and unreasonable to implement with the strict timeframe
assigned for this project. We would have to entirely change the mandelbrot 
struct so that multiple threads can run on the same mandelbrot and julia set 
concurrently. Therefore, we settled to have two threads, with one working on 
the mandelbrot set, and another working on the julia set. 

\section{Extension Testing}
Instead of hard coding all of the points of the mandelbrot set which is used
to compare to our implementation, we thought it would be reasonable to have
a concrete set of points which the program should satisfy.
\\~\\ 
There are also logical requirements the program should satisfy. 
\begin{itemize}
    \item The program should initialise correctly to display the mandelbrot
    set and julia set.
    \item The program should translate the mandelbrot set when the user inputs 
    arrow keys.
    \item The program should zoom in and out of the mandelbrot set.
    \item The program allows users to change the detail of the mandelbrot and
    julia set by choosing the number of iterations.
    \item The julia set changes depending on the point of the mandelbrot set
    currently displayed.
\end{itemize}
The testing for our Extension project was fairly easy and simple to complete,
as we can directly interact with the program, and see the outputs that it is producing.
So, if there is a problem with the calculation, then we will know instantly,
as the program will draw something that doesn't look right. This was the case
with our translate functions, as soon as we implemented them, we tested the program and
saw that it wasn't displaying quite what we intended, and so we knew that there was an error.
The main limitation of our Extension was the computation speed: drawing a mandelbrot is
highly computationally taxing, and so most implementations use a GPU to run the recursive 
functions in parallel, However due to the time limit, we could not implement a solution
that uses the GPU (CUDA library etc.), and so at high iterations, our program is quite slow.

\section{Group reflection}
Communication between group members were effective. We had a group meeting 
everyday at 13:00 which lasted on average for an hour. In this meeting, we would
discuss our progress from yesterdays task. If anyone had issues with their task,
we would attempt to solve the problem by live coding the solution. We would then
assign tasks for each member to complete by the end of the day. However,
splitting work between members could definitely be improved upon. James was tasked
to implement most of the extension of our project, due to difficulty in correctly
implementing the assemble\_sdt function in the assembler.
\\~\\
For next time, we would definitely establish communication faster and start working,
or at least planning the project much earlier.At the start of the project,
we only started talking to each other after about a week, as we were initially
all just so focused on learning the material and watching the lectures. We have
been very pressed for time in this project and so if we started a couple days earlier,
then we would definitely be able to solve almost all of the errors that we have
encountered with our current solution but don't have the time to fix.
\\~\\
Another improvement we could have made is in the task allocation, some of the errors we had with the assembler
only got fixed right at the end, and some of them are still left unresolved.
We could have been more efficient in our debugging if we all wrote the same amount of code,
as we would have the same amount of free time and would be able to use it to debug
in pairs, which is usually more efficient.
\\~\\
This also extends to our extension, as we were so pressed for time, we couldn't
all learn how to use the SDL library properly, and how to design complex functions
that recursively generate sections of a mandelbrot set. So what ended up happening
was that James did most of the extension, whilst Istvan and Prashant completed and debugged
the Assembler and the emulator.

\section{Individual reflection}
Istvan: I feel overall this was a success, especially considering we were one 
memeber down on how many we originally thought we would have. Communication 
between us was generally quite good, with the daily video call meetings as well 
as quick responsiveness from others on the Teams chat.
\\~\\
I had a bit of knowledge in makefiles and good project structure which came in
handy for compilation but I did sometimes feel like I wasn't able to make much
of an impact when it came to the actual coding. Thankfully the others were 
always ready to help and together we came up with quite a nice structure, and
I'm sure the experience from this will also help me feel like I'm pulling my own
weight more in the future.
\\~\\
Regarding the extension, I was not very well versed in Mandelbrot/Julia sets so
I could not contribute too much to that, however I feel like I gained an
understanding of the code and design choices which will help me contribute more
for other similar group projects.
\\~\\ 
James: I feel like overall we have actually done exceptionally well considering
our circumstances (only 3 people). Although our team had a fairly slow start,
once we realised that we were at a disadvantage to other teams and that the deadlines
are very tight, we organised ourselves really well and really worked hard every
day to meet the project criteria. Once a line of communication was established,
we all talked to each other regularly and helped each other with problems and with
debugging.
\\~\\
I think my main strength was my ability to decompose the problems and create logical
structures and algorithms that my team members could then use/implement. I was 
good at understanding the functions that my team members had written, and where
they might be going wrong. I would say my main two weaknesses are my lack of 
knowledge in the composition of makefiles, however Istvan was very useful in explaining
them to me, and also sometimes I found it difficult to find little errors and
inconsistencies in my own code, and so pair programming was very useful as it 
others could find the problems that I had overlooked.
\\~\\ 
Prashant: Reflecting on the project, I felt that I was an active contributor 
to the group. I communicated with everyone, and had productive conversations 
regarding issues that I ran into, and issues which they had as well.
\\~\\
I think that my main strength was that I was helpful in debugging implementations
which did not pass the test cases. I would have preferred to have more input in 
the extension, but I was not proficient enough in my knowledge of the mandelbrot
set and the SDL library. I also think I should have progressed quicker 
in coding solutions to the assembler.
\end{document}
